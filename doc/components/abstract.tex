% !TEX root = ../main.tex
% The abstract.
% Included by MAIN.TEX

\clearemptydoublepage
\phantomsection
\addcontentsline{toc}{chapter}{Abstract}

\vspace*{2cm}
\begin{center}
{\Large \textbf Abstract}
\end{center}
\vspace{1cm}



High-performance seismic wave simulators based on ADER-DG, such as SeisSol, have an inner compute kernel consisting of a chain of small sparse and dense matrix multiplications. Due to their excellent scaling characteristics, these methods benefit from any improvement of single-core performance. In the past, two code generators have been employed to produce routines optimized for each matrix product. The sparse generator unrolls the sparsity pattern into the instruction stream, while the dense generator fills in the matrix and makes optimal use of vectorization and register blocking. This work combines ideas from each in order to design a family of generators, focusing on the dense-by-sparse case, which can outperform their predecessors. It introduces UnrolledSparse, a generator which combines the sparsity pattern unrolling with a dense-style register blocking, along with GeneralSparse, which bypasses a number-of-nonzeros limit inherent in earlier sparse kernels. Other generators are developed to take advantage of regularities inside the sparsity pattern. An experiment demonstrates a speedup of 1.83, over the current dense kernel, for SeisSol's `star' matrix product. Scaling studies show that the speedup of UnrolledSparse is linear relative to the number of nonzeros in the sparse matrix. To implement these, tools for manipulating syntax trees of assembly code and abstractly traversing matrix blocks were developed.

